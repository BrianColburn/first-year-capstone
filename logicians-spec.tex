\documentclass{article}
\usepackage{titling}

\usepackage[style=numeric, backend=bibtex]{biblatex}
\addbibresource{logicians-spec.bib}

\title{Requirements Specification Document}
\author{Brandon Garcia, Brian Colburn, David Castillo, Mark Thompson, Xavier Linares}	
\date{March 2020}

\begin{document}
\setlength{\parindent}{0pt}

\begin{titlingpage}
\maketitle
\end{titlingpage}

\tableofcontents
\pagebreak

\section{Project Overview and Objectives}
\subsection{Project Overview}
Dr. Tedrow has tasked us with creating a "truth-table generator". The program is a terminal application which reads in logical statements and outputs to the user's choice of a plain-text or HTML file. The program should be written in C++ and will be used for logic/discrete math classes.


\subsection{Project Objectives}

Our objectives are as follows (in order of decreasing priority):
\begin{enumerate}
 \item Generate truth-tables
 \item Perform transformations upon statements
 \tiny \item An animated splash screen
\end{enumerate}


\section{Structured Requirements}

In outline form, demonstrate the major requirements for this project. It might be helpful to think of the highest level in your outline as the main menu for your program -- what are the key requirements/features?

\subsection{Requirement 1}
\subsubsection{Requirement 1.1}
The program must be able to parse logical statements, instantiate variables with either True or False, and evaluate the following operations:
\begin{itemize}
	\item Negation\\
		Written as ``$\sim$p".
	\item Conjunction\\
		Written as ``p AND q",\\
		``p \^{} q", or\\
		``p \& q".
	\item (Inclusive) Disjuction\\
		Written as ``p OR q",\\
		``p v q",\\
		``p V q", or\\
		``p \&\& q".
	\item Material Implication\\
		Written as ``p -$\rangle $ q".
	\item Material Equivalence\\
		Written as ``p $\langle-\rangle$ q", or\\
		``p IFF q".
\end{itemize}
Variables are defined as a letter that is not `v' (due to `v' denoting disjunction) \cite{hawthorne}.
\subsubsection{Requirement 1.2}
Describe Requirement 1.2 here!

\printbibliography

\end{document}
