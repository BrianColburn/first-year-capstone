\documentclass{article}
\usepackage{titling, lipsum}

\title{Requirements Specification Document}
\author{Rita Sperry, Blah Blah, Blah Blah, Blah Blah, Blah de Blah}	
\date{March 2020}

\begin{document}
\setlength{\parindent}{0pt}

\begin{titlingpage}
\maketitle
\end{titlingpage}

\section{Project Overview and Objectives}
\subsection{Project Overview}
Here is where I will write a brief overview about my client (who they are, what they do, etc), as well as what they would like for my team to do in the project in general terms.


\subsection{Project Objectives}

In sentence or bullet form, highlight the primary objectives/goals of the project based on your understanding of what the client wants you to accomplish.
\begin{itemize}
 \item blah
 \item blah
 \item blah
\end{itemize}


\section{Structured Requirements}

In outline form, demonstrate the major requirements for this project. It might be helpful to think of the highest level in your outline as the main menu for your program -- what are the key requirements/features?

\subsection{Requirement 1}
\subsubsection{Requirement 1.1}
Describe Requirement 1.1 here!
\subsubsection{Requirement 1.2}
Describe Requirement 1.2 here!

\section{Bibliography}

Britt, S. B. (1978). A title is written in sentence caps. \textit{Journal Title with Main Words Capitalized, 4}(2), p. 123-456.

\vspace{1em}

Sperry, R. A. (2020). The title of my article. \textit{Fancy Journal Title Here, 1}(2), p. 42-4337.

\end{document}
